\documentclass[a4paper]{article}

\usepackage{amsmath,amsfonts}

\newcommand{\R}{\mathbb{R}}

\renewcommand{\tilde}[1]{\widetilde{#1}}

\begin{document}

\section*{Documentation of \texttt{mc\_bayer\_roughbergomi\_cholesky}}
\label{sec:techn-docum-verbmc_b}

Compute the European call option price of an asset in the rough Bergomi model.

\subsection*{Usage}
\label{sec:usage}

The function takes $9$ parameters, which are explained in
Table~\ref{tab:par}. Note that the risk free interest rate is set to $r = 0$ here.

\begin{table}[!htp]
  \centering
  \begin{tabular}{c|l|c}
    Parameter & Interpretation & Range \\
    \hline
    $S_0$ & Spot price of underlying & $\R_{>0}$\\
    $\eta$ & Parameter of the stochastic volatility, ``vol-of-vol'' & $\R_{>0}$\\
    $H$ & Hurst index of the fractional Brownian motion & $]0,1/2[$\\
    $\rho$ & Correlation between Bm driving spot and volatility & $]-1,1[$\\
    $\xi$ & Spot variance & $\R_{>0}$\\
    $K$ & Strike price of the option & $\R_{>0}$\\
    $T$ & Maturity of the option (years) & $\R_{>0}$\\
    $M$ & \# of samples for Monte Carlo simultion & $\mathbb{N}$\\
    $N$ & \# of grid points for the time discretization & $\mathbb{N}$
  \end{tabular}
  \caption{Parameters of \texttt{mc\_bayer\_roughbergomi\_cholesky}}
  \label{tab:par}
\end{table}

\subsection*{Background}
\label{sec:background}

\texttt{mc\_bayer\_roughbergomi\_cholesky} computes the price of a European call option in
the rough Bergomi model by Bayer, Friz and Gatheral \cite{BFG} using Monte Carlo
simulation based on Cholesky factorization of the underlying covariance matrix.

The rough Bergomi model is described by the dynamics
\begin{gather*}
  dS_t = \sqrt{v_t} S_t dZ_t,\\
  v_t = \xi_0(t) \exp\left( \eta \tilde{W}_t - \frac{1}{2} \eta^2 t^{2H} \right),
\end{gather*}
where $W, Z$ denote two \emph{correlated} standard Brownian motions--with
correlation $\rho$--and $\eta > 0$ is interpreted as a volatility of volatility parameter. More
interestingly, for a Hurst parameter $0 < H < 1$ we have
\begin{equation*}
  \tilde{W}_t = \int_0^t K(t,s) dW_s, \quad K(t,s) = \sqrt{2H} (t-s)^{H - 1/2}.
\end{equation*}
This defines a variant of the fractional Brownian motion, sometimes called
``Riemann-Liouville fractional Brownian motion'', which is not the standard
fBm. Finally, $\xi_0$ is the forward variance curve, which is assumed to be
constant in this implementation.

Notice that samples from $\tilde{W}$ cannot be obtained by standard algorithms
for fractional Brownian motion, as $\tilde{W}$ does not have stationary increments.
However, $\tilde{W}$ is a Gaussian process, whose covariance can be computed
explicitly in terms of special functions and is given in \cite{B+}. Samples
along a grid are then provided by the Cholesky method. Note that those samples
are exact. Note that this scheme is computationally expensive.

\begin{thebibliography}{9}
\bibitem[BFG]{BFG} Ch.~Bayer, P.~Friz, J.~Gatheral: Pricing under rough
  volatility, \emph{Quantitative Finance} 16(6), 887-904, 2016.
\bibitem[B+]{B+} Ch.~Bayer, P.~K.~Friz, A.~Gulisashvili, B.~Horvath,
  B.~Stemper: \emph{Short-time near-the-money skew in rough fractional
    volatility models}, arXiv preprint arXiv:1703.05132, 2017.
\end{thebibliography}

\end{document}

%%% Local Variables:
%%% mode: latex
%%% TeX-master: t
%%% End:
